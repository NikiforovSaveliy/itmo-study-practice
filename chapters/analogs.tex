\chapter{Сравнение аналогов проектируемой системы}
    В ходе тщательного анализа и изучения предметной области, %
    был проведен обширный поиск и исследование существующих аналогов %
    на рынке. Данное исследование включало в себя изучение различных приложений и %
    сервисов, их функционала, возможностей и особенностей. %
    В результате, был получен ряд ключевых аналогов, %
    которые в настоящее время существуют и активно используются. %
    Эти аналоги представляют собой различные продукты и решения, %
    которые в той или иной степени могут совпадать с поставленными целями %
    и задачами. Изучение данных аналогов поможет лучше проанализировать %
    существующий рынок, определить ключевые тренды и возможности для %
    инноваций. Изученные аналоги изображены в таблице \ref{table:kysymys}.
    
    \newpage
    \begin{landscape}
        
        \begin{table}[]
            \caption{Таблица сравнения аналогов}
            \resizebox{\columnwidth}{!}{%
            \begin{tabular}{|p{9em}|p{13em}|p{9em}|p{10em}|p{9em}|}
            
            \hline
            Название Аналога & Возможности & Тариф & Достоинства & Недостатки \\
            \hline
            1 & 2 & 3 & 4 & 5 \\
            \hline
            Medisafe & Мобильное приложение, которое помогает отслеживать приём лекарств по дате и времени приёма с примечанием. Также существует интеграция с приложением хранящим и обновляющим данные о здоровье и есть возможность составлять отчёт о приёме. & Бесплатное досупны базовые функции, остальные как кастомизация уведомлений и внешнего вида и отсутсвие рекламы доступны по подписке 4.99\$ в месяц и 39.99\$ в год. & Возможность следить за приёмом лекарств членов семьи и создание отчётности о лечении. Присутсвие примечения о способе приёма лекарства и дозировки. Интерграция с приложением следящим за показателями здоровья. & Система уведомлений не гибкая и не настраиваемая. \\
            \hline
            Mytherapy & Мобильное приложение, которое напоминает о приёме лекарств, отслеживает показатели здоровья и имеет возможность создания отчётов приёме. & Бесплатное приложение & Существование журнала о пропущенных и подтверждённых приёмов лекарств. Отслеживание количества лекарств. Поддержка схем дозировок. Выбор измерений показателей взависимости от заболевания. & Обязует следить за количеством лекарств. Не удобный выбор временит приёма. \\
            \hline
        \end{tabular}%
            }\label{table:kysymys}
            \end{table}
        
        \newpage

        \begin{table}[]
            \begin{flushright}
                Продолжение таблицы 1.1
            \end{flushright}
            \resizebox{\columnwidth}{!}{%
            \begin{tabular}{|p{9em}|p{13em}|p{9em}|p{10em}|p{9em}|}
            \hline
            1 & 2 & 3 & 4 & 5 \\
            \hline
            Мои таблетки & Мобильное приложение напоминающее о приёме лекарств. Так же имеющие возможность хранить историю приёмов. И создавать примечание о приёме. & Бесплатно доступно 10 курсов приёмов лекарств. По подписке 1.99\$ в месяц доступно неограниченное количество курсов приёмов. & Гибкая система уведомлений. Архив приёмов лекарств. & Отсутсвие семейнного доступа и учёта пропущенных приёмов. \\
            \hline
            Mr.Pillster & Мобильное приложение напоминающие о приёме лекарств и о измерении данных о здоровье. Существует возможность создавать график приёмов и измерений. Семейный доступ к приёмам лекарств. Быстрые кнопки “принять” и “отложить” в пуш-уведомлении & Бесплатно доступно добавление только прёмов лекарств. По подписке 329 рублей раз в три месяца, 749 рублей в год и 1790 рублей единоразово доступны все функции. & Возможность создать напоминание о измерении данных о здоровье. Семейный доступ к приёму лекарств. Возможность отложить приём лекарства при появлении уведомления. & В сравнение с другими приложениями не полный список типов дозировок. Также отсутсвие возможности создать собственное примечание к приёму лекарства. \\
            \hline
            Pills time & Мобильное для напоминаний о приёме лекарств и приёмах к врачу. Семейный доступ к приёмам. Возможность добавить фото к добавленному лекарству. Гибкий выбор периодичности приёма. & Бесплатное добавление курса лекарств, без. По подписке 99 рублей в месяц в год будут доступны все функции. & Большой список примечаний связанных с едой. Семейный доступ. Возможность создать расписание приёмов у врача. & Неудобная система уведомлений. Короткий список типов дозировок. \\
            \hline 
        \end{tabular}%
            }
            \end{table}
        
    \end{landscape}%

    В сравнении аналогов, были рассмотрены несколько ключевых аспектов, которые могут быть важными для пользователей, а именно:

    \begin{enumerate}
        \item Функциональность.
        \begin{itemize}
            \item Mediasafe предоставляет широкий спектр функций, включая возможность создания индивидуального графика лечения, отслеживание прогресса и интеграцию с другими приложениями;
            \item Mytherapy предоставляет следующий набор функций: составление графика лечения, гибкую настройку уведомлений, семейный доступ;
            \item Мои таблетки, в отличии от других аналогов, не имеет семейного доступа, но имеет гибкую систему настройки уведомлений и архив приема лекарств;
            \item Mr.Pillster предлагает более ограниченный набор функций, включая создание графика лечения и семейный доступ;
            \item Pills time предоставляет только базовые функции, такие как создание графика лечения, возможность добавления заметок и семейный доступ;
        \end{itemize}
    
        \item Удобство использования.
        \begin{itemize}
            \item Mediasafe имеет интуитивно понятный интерфейс и простую навигацию, что делает его удобным для пользователей;
            \item Mytherapy имеет простой интерфейс, но некоторые функции могут быть неудобными в использовании;
            \item Mr.Pillster имеет удобный интерфейс, но сложную навигацию;
            \item Мои таблетки имеет интуитивно понятный интерфейс и простую навигацию, что делает его удобным для пользователей;
            \item Pills time имеет сложный интерфейс, который может быть трудным для понимания и использования;
        \end{itemize}
    
        \item Стоимость.
        \begin{itemize}
            \item Mediasafe предоставляет бесплатный доступ к основным функциям, остальные доступны по подписке 4.99 \$ в месяц;
            \item Mytherapy - бесплатное приложение;
            \item Мои таблетки предоставляет доступ к 10 бесплатным курсам лечения, а для неограниченного доступа необходима подписка 1.99 \$ в месяц;
            \item Mr.Pillster предоставляет бесплатный доступ к добавлению приемов лекарств, для разблокирования других функций необходима подписка, которая стоит 3.6 \$ раз в 3 месяца;
            \item Pills time предоставляет бесплатный доступ к добавлению приемов лекарста, для разблокирования других функций необходима подписка 1 \$ в месяц;
        \end{itemize}
            
    \end{enumerate}
    
    
    Исходя из сравнения, можно сделать вывод, что аналог Mytherapy является наиболее предпочтительным выбором для пользователей, которые ищут веб-приложение для составления графика лечения на основании рекомендаций врача. Он предлагает широкий спектр функций, удобный интерфейс и польностью бесплатное использование, поэтому, при дальнейшем анализе, стоит опираться на этот аналог.
    