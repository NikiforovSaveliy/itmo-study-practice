\chapter{Анализ функицональных и нефукциональных требований}
    В данной главе будет представлен анализ будущего приложения %
    с точки зрения выделения функциональных и нефукциональных требований.
    \section{Функциональные требования}
        В процессе анализа требований к веб-приложению для %
        составления графика лечения на основе рекомендаций врача были выделены %
        следующие фукнциональные требовация, которые необходимы для полноценного %
        функционирования приложения:
        \newline
        \begin{enumerate}
            \item Веб-приложение должно позволять пациенту или его %
            лечащему врачу создавать график лечения на основании %
            рекомендаций и назначений. Для этого необходимо предоставить %
            пользователю возможность добавлять информацию о лекарствах, %
            дозах и промежутках времени приема, а также продолжительность %
            лечения. График должен быть составлен на необходимый период %
            времени и сохраняться для каждого пациента.
            
            \item Приложение должно иметь функцию добавления %
            рекомендаций и назначений, включая лекарства, %
            дозы и промежутки времени приема.

            \item Приложение должно позволять врачу указывать %
            продолжительность лечения, чтобы график был составлен %
            на необходимый период времени.

            \item Приложение должно предоставлять возможность %
            сохранять и редактировать график лечения для каждого %
            пациента, чтобы врач мог обновлять его при необходимости.

            \item Приложение должно иметь функцию проверки наличия %
            противопоказаний и взаимодействия между различными %
            лекарствами, чтобы врач мог быть уверен в безопасности %
            назначенного лечения.

            \item Приложение должно предоставлять возможность %
            врачу проверить, были ли выполнены назначения пациентом, %
            путем отметки о выполнении каждой рекомендации.

            \item Приложение должно создавать напоминания для пациента %
            о приеме лекарств и выполнении других назначений, %
            которые были добавлены самим пациентом или его лечащим врачом.

            \item Приложение должно обеспечивать безопасное хранение %
            конфиденциальной информации о пациентах и доступ только %
            для авторизованных врачей.
        \end{enumerate}

    \section{Нефункциональные требования}

        После формулирования функциональных требований к системе и анализа %
        системы с точки зрения взаимодействия с конечным потребителем, были выделены %
        следующие нефункциональные требования:
        \newline
        \begin{enumerate}
            \item Веб-приложение должно иметь интуитивно понятный и %
            привлекательный интерфейс для пользователей всех возрастов %
            и уровней навыков. 

            \item Приложение должно поддерживать различные устройства и %
            разрешения экранов, включая мобильные устройства, планшеты %
            и компьютеры. 

            \item Гарантия конфиденциальности и безопасности %
            персональных медицинских данных пациентов. Так как данный аспект %
            строго регулируется государством. 

            \item Обеспечить совместимость с различными современными бразуерми, %
            при помощи который будет исполняться взаимодействие с приложением для %
            удобства пользователей.

            \item Обеспечить быструю загрузку и отзывчивость интерфейса %
            при работе с графиком лечения и рекомендациями врача.

            \item Поддерживать возможность локализации интерфейса на %
            русском языке для удовлетворения потребностей пользователей %
            на территории Российской Федерации.

            \item Реализовать авторизацию и управление доступом к %
            приложению и данным для обеспечения безопасности и %
            ограничения доступа к конфиденциальным данным.

            \item Обеспечить отказоустойчивость для минимизации времени %
            простоя и потери данных в случае сбоев или неполадок. %
            Приложение должно быть обеспечено необходимыми механизмами %
            для минимизации времени простоя и потери данных в случае %
            возникновения сбоев или неполадок. Это означает, что %
            система должна быть способна восстанавливаться после %
            сбоев без значительного воздействия на функциональность и производительность, обеспечивая непрерывность предоставления услуг пользователям.
        \end{enumerate}