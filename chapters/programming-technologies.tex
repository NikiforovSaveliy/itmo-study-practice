\chapter{Технологии программирования}
    В данной главе будут представлены выбранные технологии программироания, %
    которые будут применены при разработке веб-приложения для составления %
    графика лечения на основании рекомендаций врача.

    \section{Веб-приложение}
        Для реализации клиент-серверной архитектуры будут применяться %
        следующие технологии программирования:
        \begin{enumerate}
            \item Python - \cite{python} это интерпретируемый язык программирования, который %
            широко используется в веб-разработке. Он имеет простой и понятный синтаксис, %
            множество библиотек для создания веб-приложений и хорошо масштабируется. %
            \item Django \cite{django} - это обширная библеотека для %
            языка Python, которая предоставляет множество инструментов для %
            разработки веб-приложений. Она обеспечивает высокую производительность, %
            безопасность и масштабируемость. Django также имеет встроенную поддержку %
            для работы с базами данных, шаблонизации и аутентификации.
            \item JavaScript \cite{javascript} - это интерпретируемый язык программирования, %
            который широко используется %
            для разработки веб-приложений. Также нативно поддерживается современными браузерами.
            \item React \cite{react} - это библиотека для создания пользовательских %
            интерфейсов на языке JavaScript. Она предоставляет мощные инструменты %
            для создания интерактивных и высокопроизводительных веб-приложений. %
            React позволяет легко управлять состоянием приложения и обновлять %
            его части, что делает его идеальным выбором для разработки %
            сложных интерфейсов.
        \end{enumerate}
    
    \section{Система уведомлений пользователей}
        Для реализации системы уведомлений пользователей было принято решение %
        воспользоваться телеграмм-ботом для рассылки уведомлений. Данный выбор связан %
        с обширной базой пользователей телеграма\cite{telegram}.

        Для программной реализации данной концепции была выбрана библиотека Aiotelegram \cite{aiotelegram} %
        Данная библиотека предназначена для работы с интерфейсом программирования приложения (API) Telegram. Она позволяет %
        разработчикам создавать ботов для Telegram и применять интерактивные пути взаимодействия с пользователем. В данном %
        случае будет создан телеграмм-бот для отправки уведомлений пациентам.
    
    \section{Развертывание приложения}
        Для успешной эксплуатации приложения, планируется использовать современные технологии %
        развертывания приложений. Для достижения гибкости использования предоставленной провайдером %
        серверной инфраструктуры будет использоваться инструмент контейнеризации приложений под названием %
        Docker \cite{docker}. Так как приложения является небольшим, то для оркестрации данных контейнеров будет использоваться %
        поставляемый вместе с Docker инструмент Docker-compose \cite{docker-compose}.
        Использование данного подхода технологий позволит создать надежное и %
        масштабируемое веб-приложение для составления графика лечения на %
        основании рекомендаций врача. 