\intro 
    В современном мире все больше людей обращаются к веб-сервисам для %
    автоматизации и упрощения собственной жизни. Они органично вписались %
    во все направления жизни людей современного общества.  

    Одним из таких инструментов упрощения и автоматизации жизнедеятельности может %
    стать веб-приложение для составления графика лечения на основании рекомендаций %
    врача. Помимо удобства для пациентов, данное приложение может помочь докторам %
    более эффективно контроллировать процесс лечения.

    Актуальность приложения обоснована следующим: по данным исследования Ассоциации директоров по коммуникациям и %
    корпоративным медиа России (АКМР)\cite{internet_users} от 2023 года показывает что 78 процентов %
    Россиян пользуются интернетом на регулярной основе, что показывает большой %
    охват аудитории для полноценной реализации и интеграции будущего веб-приложения.

    В дополнение к этому, с учетом увеличивающейся численности старшего поколения %
    и необходимости улучшения доступа к медицинским услугам, веб-приложение %
    подобного рода может иметь значительный потенциал для улучшения уровня %
    здравоохранения. Это также может помочь уменьшить нагрузку на медицинский %
    персонал, освободив их время для более критически важных задач. Кроме того, %
    с учетом текущей ситуации в мире, веб-приложения для здравоохранения открывают %
    новые возможности для дистанционного взаимодействия и контроля за процессом %
    лечения, что делает их еще более актуальными и востребованными.

    Целью данной практической работы является проектирование и разработка веб-приложения для составления графика лечения на %
    основании рекомендаций врача.

    Из составленной цели были выдвинуты следующие задачи:
        \begin{itemize}
            \item Анализ предметной области;
            \item Анализ конкурентов и аналогов приложения;
            \item Составление функциональных и не функциональных требований;
            \item Разработка базы данных приложения;
            \item Составление диаграм взаимодействия пользователей с системой;
            \item Выбор технологий программирования для создания конечного продукта;
            \item Создание дизайн-макет приложения; 
        \end{itemize}